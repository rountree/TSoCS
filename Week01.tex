\documentclass{tufte-handout}

\title{Week 1:  The Science of Computer Science}

\author{Barry Rountree}
\date{Spring 2019}
%\date{28 March 2010} % without \date command, current date is supplied

%\geometry{showframe} % display margins for debugging page layout

\usepackage{graphicx} % allow embedded images
  \setkeys{Gin}{width=\linewidth,totalheight=\textheight,keepaspectratio}
  \graphicspath{{graphics/}} % set of paths to search for images
\usepackage{amsmath}  % extended mathematics
\usepackage{booktabs} % book-quality tables
\usepackage{units}    % non-stacked fractions and better unit spacing
\usepackage{multicol} % multiple column layout facilities
\usepackage{lipsum}   % filler text
\usepackage{fancyvrb} % extended verbatim environments
  \fvset{fontsize=\normalsize}% default font size for fancy-verbatim environments

% Standardize command font styles and environments
\newcommand{\doccmd}[1]{\texttt{\textbackslash#1}}% command name -- adds backslash automatically
\newcommand{\docopt}[1]{\ensuremath{\langle}\textrm{\textit{#1}}\ensuremath{\rangle}}% optional command argument
\newcommand{\docarg}[1]{\textrm{\textit{#1}}}% (required) command argument
\newcommand{\docenv}[1]{\textsf{#1}}% environment name
\newcommand{\docpkg}[1]{\texttt{#1}}% package name
\newcommand{\doccls}[1]{\texttt{#1}}% document class name
\newcommand{\docclsopt}[1]{\texttt{#1}}% document class option name
\newenvironment{docspec}{\begin{quote}\noindent}{\end{quote}}% command specification environment

\begin{document}

\maketitle% this prints the handout title, author, and date

\section{Live Ubuntu}

I have created a custom Ubuntu image\footnote{\url{URL to come}} that will be slightly
easier to work with.  There is no persistent storage, however,
so unless you have off your work elsewhere, you'll lose it as
soon as you log out. Saving your work off to a remote repository will probably be the
easiest solution.  You might also be able to use a second USB, scp, etc.

To boot off a USB in OSX, put in the USB, hold down on the alt/option key, and reboot.
You should be given the choice of an ``EFI Boot'' device.  Choose that.

I assume networking will not be a problem, but we will figure that out once we get into
the lab.  Choosing the network at the EFI screen is not necessary (at least not on my machine).

When the grub screen appears (``Try Ubuntu without installing''), you can either select that or,
if you would like a bit more performance, hit `e'.  This allows you to edit the boot 
script.  Adding ``toram'' (no quotes) before the ``---'' will load the entire USB image into
RAM.  Responsiveness should be much better, but you won't have as much space for other things.
Boot time will also take a couple of minutes.  If you go this route, hit F10 to continue the
boot process.


\section{byobu/tmux}

Most of our time will be spent in the terminal, often editing multiple files with another
window open for running programs.  One way to handle this is multiple windows.  Another is 
multiple tabs.  A third is running a terminal multiplexor like tmux or byobu (byobu is what
I use; the two are mostly indistiguishable).  Briefly:  \texttt{Ctrl-a c} creates a new
window, \texttt{Ctrl-a n} goes to the next window, \texttt{Ctrl-a p} goes to the previous
window, \texttt{Ctrl-a 3} goes to window 3, and most useful, \texttt{Ctrl-a A} allows you to
label a window.  

\section{Intel Documentation}




\section{Firestarter}

\section{R}

\section{\LaTeX}

\section{Bash shell scripting}

\section{Modprobe and the msr kernel module}

\section{rdmsr and wrmsr command line tools}

\section{

\end{document}
