\documentclass{tufte-handout}

\title{Week 1:  The Science of Computer Science}

\author{Barry Rountree}
\date{Spring 2019}
%\date{28 March 2010} % without \date command, current date is supplied

%\geometry{showframe} % display margins for debugging page layout

\usepackage{graphicx} % allow embedded images
  \setkeys{Gin}{width=\linewidth,totalheight=\textheight,keepaspectratio}
  \graphicspath{{graphics/}} % set of paths to search for images
\usepackage{amsmath}  % extended mathematics
\usepackage{booktabs} % book-quality tables
\usepackage{units}    % non-stacked fractions and better unit spacing
\usepackage{multicol} % multiple column layout facilities
\usepackage{lipsum}   % filler text
\usepackage{fancyvrb} % extended verbatim environments
  \fvset{fontsize=\normalsize}% default font size for fancy-verbatim environments

% Standardize command font styles and environments
\newcommand{\doccmd}[1]{\texttt{\textbackslash#1}}% command name -- adds backslash automatically
\newcommand{\docopt}[1]{\ensuremath{\langle}\textrm{\textit{#1}}\ensuremath{\rangle}}% optional command argument
\newcommand{\docarg}[1]{\textrm{\textit{#1}}}% (required) command argument
\newcommand{\docenv}[1]{\textsf{#1}}% environment name
\newcommand{\docpkg}[1]{\texttt{#1}}% package name
\newcommand{\doccls}[1]{\texttt{#1}}% document class name
\newcommand{\docclsopt}[1]{\texttt{#1}}% document class option name
\newenvironment{docspec}{\begin{quote}\noindent}{\end{quote}}% command specification environment

\begin{document}

\maketitle% this prints the handout title, author, and date

\section{Live Ubuntu}

I have created a custom Ubuntu image\footnote{\url{https://github.com/rountree/TSoCS-usb}} that will be slightly
easier to work with.  There is no persistent storage, however,
so unless you have off your work elsewhere, you'll lose it as
soon as you log out. Saving your work off to a remote repository will probably be the
easiest solution.  You might also be able to use a second USB, scp, etc.

To boot off a USB in OSX, put in the USB, hold down on the alt/option key, and reboot.
You should be given the choice of an ``EFI Boot'' device.  Choose that.

I assume networking will not be a problem, but we will figure that out once we get into
the lab.  Choosing the network at the EFI screen is not necessary (at least not on my machine).

When the grub screen appears (``Try Ubuntu without installing''), you can either select that or,
if you would like a bit more performance, hit `e'.  This allows you to edit the boot 
script.  Adding ``toram'' (no quotes) before the ``---'' will load the entire USB image into
RAM.  Responsiveness should be much better, but you won't have as much space for other things.
Boot time will also take a couple of minutes.  If you go this route, hit F10 to continue the
boot process.

\section{byobu/tmux}

Most of our time will be spent in the terminal, often editing multiple files with another
window open for running programs.  One way to handle this is multiple windows.  Another is 
multiple tabs.  A third is running a terminal multiplexor like tmux or byobu (byobu is what
I use; the two are mostly indistiguishable).  Briefly:  \texttt{Ctrl-a c} creates a new
window, \texttt{Ctrl-a n} goes to the next window, \texttt{Ctrl-a p} goes to the previous
window, \texttt{Ctrl-a 3} goes to window 3, and most useful, \texttt{Ctrl-a A} allows you to
label a window.  

\section{Github}

Syllabus and weekly notes and assignments are at github.\footnote{\url{https://github.com/rountree/TSoCS}}

We'll have a very brief introduction on how to set up your own repo, add files, save them back to Github,
and pull them down again elsewhere.

\section{\LaTeX}

A very brief introduction to \LaTeX workflows, including how to embed graphs into
\LaTeX documents.

\section{The \texttt{msr} kernel module}

We will add the \texttt{msr} kernel module into the running kernel and set the permissions of the device
files that result using \texttt{chown}.

\section{\texttt{rdmsr} and \texttt{wrmsr}}

Download\footnote{\url{https://01.org/msr-tools}} and compile these utilities.

\section{Intel Documentation}

We'll start from the main page for the Sofware Developers' Manual.\footnote{\url{https://software.intel.com/en-us/articles/intel-sdm}}
For these assignments, we will spend most of our time in Chapter 14 of volume 3B and chapter 2 of volume 4.
Intel provides four different formats:  all 4 volumes together, 4 volumes across 4 files, and 4 volumes across 10 files.
Particularly when you're doing searches, the smaller documents are easier to work with.

\section{Bash shell scripting}

Do a quick sanity check to make sure we can read the time stamp counter using MSRs.  Get a bit of data.

\section{R}

Pull that data into R and graph it.  Save it to a \texttt{png} file.  At what rate does the TSC increment?

\section{Firestarter}

Pull down and compile Firestarter.\footnote{\url{https://github.com/tud-zih-energy/FIRESTARTER}}  
Note that the build process is a little nonstandard. 


\section{A few questions to consider}
\begin{enumerate}
	\item Section 14.2 of Volume 3B describes two hardware counters APERF and MPERF.  At what rate do these counters increment during the sleep test?
	\item Define \textit{effective clock frequency} as $\frac{\delta\texttt{MPERF}}{\delta\texttt{APERF}}$. What is the effective clock frequency during sleep?
	\item What is the effective clock frequency during a 10-second run of Firestarter on a single core?  On all cores?
	\item Use \texttt{setitimer}\footnote{Run \texttt{man setitimer} for details} to create a simple program that prints ``Hello, world'' every second.
	\item Tear down the \texttt{rdmsr} utility to see how to access MSRs via C.  Add the code to the program above to let you read \texttt{APERF} and \texttt{MPERF} once per millisecond.  Graph the results.
\end{enumerate}

\section{Assignment}
Modify Firestarter to report back the effective clock frequency at multiple intervals give a fixed amount of work.  Create two graphs:  one showing wall-clock performance versus sampling rate (Does sampling more often slow down Firestarter?  How bad is it?) and a graph that shows effective clock frequency over time.  Repeat this for at least two configurations:  running all cores and running a single core.





\end{document}
