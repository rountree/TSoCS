\documentclass{tufte-handout}

\title{Syllabus:  The Science of Computer Science}

\author{Barry Rountree}
\date{Spring 2019}
%\date{28 March 2010} % without \date command, current date is supplied

%\geometry{showframe} % display margins for debugging page layout

\usepackage{graphicx} % allow embedded images
  \setkeys{Gin}{width=\linewidth,totalheight=\textheight,keepaspectratio}
  \graphicspath{{graphics/}} % set of paths to search for images
\usepackage{amsmath}  % extended mathematics
\usepackage{booktabs} % book-quality tables
\usepackage{units}    % non-stacked fractions and better unit spacing
\usepackage{multicol} % multiple column layout facilities
\usepackage{lipsum}   % filler text
\usepackage{fancyvrb} % extended verbatim environments
  \fvset{fontsize=\normalsize}% default font size for fancy-verbatim environments

% Standardize command font styles and environments
\newcommand{\doccmd}[1]{\texttt{\textbackslash#1}}% command name -- adds backslash automatically
\newcommand{\docopt}[1]{\ensuremath{\langle}\textrm{\textit{#1}}\ensuremath{\rangle}}% optional command argument
\newcommand{\docarg}[1]{\textrm{\textit{#1}}}% (required) command argument
\newcommand{\docenv}[1]{\textsf{#1}}% environment name
\newcommand{\docpkg}[1]{\texttt{#1}}% package name
\newcommand{\doccls}[1]{\texttt{#1}}% document class name
\newcommand{\docclsopt}[1]{\texttt{#1}}% document class option name
\newenvironment{docspec}{\begin{quote}\noindent}{\end{quote}}% command specification environment

\begin{document}

\maketitle% this prints the handout title, author, and date

\section{Particulars}
\begin{tabular*}{0.75\textwidth}{l r}
	Date and Time:  & 1p -- 3:40p Friday	\\
	Room		& TBD			\\
	Final Exam	& TBD			\\
	Office Hours	& 3:40p -- 5p Friday	\\
	email		& \texttt{rountreb@sonoma.edu} \\
			& \texttt{barry.rountree@protonmail.com} \\
\end{tabular*}

\section{Prerequisites}
I recommend students have taken at least two of Operating Systems, Compiler and
Computer Architecture.  Familiarity with bash, C and assembly will be helpful.

\section{Textbooks and Materials}
I don't require the purchase of any textbooks.  We will be using primarily online,
freely-available documentation.  I strongly recommend having a free 
github\footnote{\url{http://www.github.com}} account;
a paid version will allow anonymity, as will the use of a pseudonym.  I also 
recommend a 4GB+ USB drive; I will have several to loan.  Having root access on a 
linux machine will be helpful but is not required.

\section{Grading}
Grading will be based on weekly assignments and a final exam.  The final will count
for no more than 30\% of the final grade.  I will not accept late work.  Assignments
are due in my email at 8p on the following Friday (you have a little over one week,
and two class periods, to work on them).  As we are going to be doing almost entirely
experimental work, I will be grading assignment on what you learned (and documented),
rather than the correctness and polish of the final product.

\section{Collaboration}
Please do.  You are expected to share ideas, code, results and debugging tips 
\textit{so long as you give credit where credit is due}. For example, if you were
planning on making use of a certain processor feature and realized at the last 
moment that the computer you have access to does not have the right processor,
you make ask (nicely) to duplicate the dataset of another student \textit{and explicitly
note that you have done so}.  In this case, a couple of paragraphs describing
how you determined the processor feature set will be helpful for your grade.
The (hopefully) more common case will be a group of students working together.  
Each will be turning in their own copy, and each copy lists all the contributors.

Using ideas, code and results without permission and acknowledgement will lead,
at a minimum, to failing the assignment and may lead to failing the class as well as
other university sanctions.  


\end{document}
